{
 "cells": [
  {
   "cell_type": "markdown",
   "id": "1f03e87a",
   "metadata": {},
   "source": [
    "\\documentclass{article}\n",
    "\n",
    "% Language setting\n",
    "% Replace `english' with e.g. `spanish' to change the document language\n",
    "\\usepackage[english]{babel}\n",
    "\n",
    "% Set page size and margins\n",
    "% Replace `letterpaper' with `a4paper' for UK/EU standard size\n",
    "\\usepackage[letterpaper,top=2cm,bottom=2cm,left=3cm,right=3cm,marginparwidth=1.75cm]{geometry}\n",
    "\n",
    "% Useful packages\n",
    "\\usepackage{amsmath}\n",
    "\\usepackage{graphicx}\n",
    "\\usepackage[colorlinks=true, allcolors=blue]{hyperref}\n",
    "\n",
    "\\title{Técnicas de Agrupamiento aplicado a analizar la siniestralidad de una aseguradora }\n",
    "\\author{Irwinng Cabrera Rodríguez}\n",
    "\\date{Noviembre 2025}\n",
    "\n",
    "\\begin{document}\n",
    "\\maketitle\n",
    "\n",
    "\\section{Introducción}\n",
    "\n",
    "El análisis de agrupamiento también conocido como \\textit{clustering} es una técnica de aprendizaje no supervisado que se utiliza para descubrir grupos o patrones ocultos dentro de un conjunto de datos. El propósito del análisis de agrupamiento es encontrar patrones naturales en los datos, resumir grandes volúmenes de información y identificar segmentos, perfiles o comportamientos similares.\n",
    "\n",
    "Para este trabajo se analiza el comportamiento del Grupo GZ que cuenta con 38 observaciones, las cuales son variables numéricas que describen el comportamiento de la grupo analizar.\n",
    "\n",
    "Para este análisis se aplicaran metodología como $K$-Medias para la determinación del número de grupos.\n",
    "\n",
    "\\section{Previo}\n",
    "\n",
    "Un \\textit{seguro de auto} es un contrato entre una aseguradora y el cliente que tiene el objetivo de protegerte su vehículo en cuestión económica en caso de que ocurra un accidente, robo u otro daño relacionado con tu vehículo.\n",
    "\n",
    "\\subsection{¿Cómo funciona?}\n",
    "\n",
    "Tú pagas una prima y, a cambio, la aseguradora se compromete a cubrir ciertos gastos según la póliza que contrates.\n",
    "\n",
    "\\subsection{Conceptos clave en una póliza de seguros}\n",
    "\n",
    "\\begin{enumerate}\n",
    "    \\item \\textbf{Asegurado:} La persona o entidad que recibe la protección del seguro.\n",
    "    \\item \\textbf{Contratante:} Quien compra y paga la póliza.\n",
    "    \\item \\textbf{Beneficiario:} Persona que recibe la indemnización en caso de siniestro.\n",
    "    \\item \\textbf{Aseguradora:} Empresa que cubre económicamente en caso de un accidente.\n",
    "    \\item \\textbf{Prima:} El costo del seguro.\n",
    "    \\item \\textbf{Coberturas:} Todo lo que sí está protegido por el seguro.\n",
    "    \\item \\textbf{Exclusiones:} Situaciones que no están cubiertas.\n",
    "    \\item \\textbf{Suma asegurada:} El límite máximo que la aseguradora pagará.\n",
    "    \\item \\textbf{Deducible:} Cantidad que pagas antes de que el seguro cubra el resto.\n",
    "    \\item \\textbf{Vigencia:} Periodo durante el cual la póliza está activa.\n",
    "    \\item \\textbf{Siniestro:} Evento que activa el seguro.\n",
    "    \\item \\textbf{Indemnización:} Pago o beneficio otorgado por la aseguradora.\n",
    "    \\item \\textbf{Condiciones generales:} Reglas y definiciones de la póliza.\n",
    "    \\item \\textbf{Endoso:} Modificación a la póliza después de contratarla.\n",
    "\\end{enumerate}\n",
    "\n",
    "\n",
    "\\section{Metodología}\n",
    "\n",
    "El conjunto de datos se compone de variables categóricas y numéricas que describen las caracteríscticas de la cuenta GZ.\n",
    "\n",
    "Los datos los trabajamos con el programa \\texttt{Python} con la función \\texttt{StandardScaler}, después, se trabaja con el algoritmo OPTICS (agrupamiento jerárquico).\n",
    "\n",
    "\\subsection{OPTICS (agrupamiento jerárquico)}\n",
    "\n",
    "OPTICS es una extensión del algoritmo DBSCAN, por lo cual, OPTICS genera una estructura jerárquica de clústeres, donde puedes ver cómo los grupos se forman y se dividen al variar la densidad.\n",
    "\n",
    "\\textbf{Ventajas}\n",
    "\\begin{enumerate}\n",
    "\\item Detecta clústeres de distinta densidad (DBSCAN no puede).\n",
    "\\item Identifica ruido y puntos atípicos automáticamente (label = -1).\n",
    "\\item Produce una estructura jerárquica de clústeres\n",
    "\\end{enumerate}\n",
    "\n",
    "\\textbf{Desventajas}\n",
    "\\begin{enumerate}\n",
    "\\item Ds más lento que $K$-Medias o DBSCAN\n",
    "\\item Más difícil de interpretar\n",
    "\\end{enumerate}\n",
    "\n",
    "\\section{Error Cuadrático Medio (MSE)}\n",
    "\n",
    "\\textbf{¿Qué es?} \\\\\n",
    "El error cuadrático medio mide entre los valores reales y los valores predichos.\n",
    "\n",
    "\\textbf{Fórmula:}\n",
    "\\begin{equation}\n",
    "MSE = \\frac{1}{n} \\sum_{i=1}^{n} (y_i - \\hat{y}_i)^2\n",
    "\\end{equation}\n",
    "\n",
    "\\textbf{Interpretación:}\n",
    "\\begin{itemize}\n",
    "    \\item Valores más cercanos a 0 indican mejor desempeño.\n",
    "    \\item Es sensible a valores atípicos.\n",
    "\\end{itemize}\n",
    "\n",
    "\\subsubsection*{Raíz del Error Cuadrático Medior (RMSE)}\n",
    "\n",
    "\\textbf{¿Qué es?} \\\\\n",
    "Es la raíz del error cuadrático medio y Permite interpretar el error en las mismas unidades que la variable objetivo.\n",
    "\n",
    "\\textbf{Fórmula:}\n",
    "\\begin{equation}\n",
    "RMSE = \\sqrt{MSE}\n",
    "\\end{equation}\n",
    "\n",
    "\\textbf{Interpretación:}\n",
    "\\begin{itemize}\n",
    "    \\item Indica en promedio cuánto se equivoca el modelo.\n",
    "    \\item Es útil para comparar con magnitudes reales de la variable.\n",
    "\\end{itemize}\n",
    "\n",
    "\\subsubsection*{Error Absoluto Medio (MAE)}\n",
    "\n",
    "\\textbf{¿Qué es?} \\\\\n",
    "Mide el error absoluto medio entre la predicción y el valor real. \n",
    "\n",
    "\\textbf{Fórmula:}\n",
    "\\begin{equation}\n",
    "MAE = \\frac{1}{n} \\sum_{i=1}^{n} | y_i - \\hat{y}_i |\n",
    "\\end{equation}\n",
    "\n",
    "\\textbf{Interpretación:}\n",
    "\\begin{itemize}\n",
    "    \\item Es más robusto frente a valores atípicos.\n",
    "    \\item Indica cuánto se equivoca el modelo en promedio.\n",
    "\\end{itemize}\n",
    "\n",
    "\\subsubsection*{Coeficiente de Determinación ($R^2$)}\n",
    "\n",
    "\\textbf{¿Qué es?} \\\\\n",
    "Mide qué proporción de la variabilidad de la variable objetivo es explicada por el modelo.\n",
    "\n",
    "\\textbf{Fórmula:}\n",
    "\\begin{equation}\n",
    "R^2 = 1 - \\frac{\\sum_{i=1}^{n} (y_i - \\hat{y}_i)^2}{\\sum_{i=1}^{n} (y_i - \\bar{y})^2}\n",
    "\\end{equation}\n",
    "\n",
    "\\textbf{Interpretación:}\n",
    "\\begin{itemize}\n",
    "    \\item Valores cercanos a 1 indican un excelente ajuste.\n",
    "    \\item Valores cercanos a 0 indican que el modelo no explica la variabilidad.\n",
    "    \\item Puede tomar valores negativos si el modelo es peor que predecir el promedio.\n",
    "\\end{itemize}\n",
    "\n",
    "\n",
    "\\section{Resultados}\n",
    "\n",
    "\\begin{figure}\n",
    "    \\centering\n",
    "   \\includegraphics[width=0.95\\textwidth]{imgs/alcance_ordenado.png}\n",
    "    \\caption{Gráfico de alcance ordenado - OPTICS}\n",
    "    \\label{fig:alcance_ordenado}\n",
    "\\end{figure}\n",
    "\n",
    "\\subsection{Gráfica de distancia de alcance}\n",
    "\n",
    "Como se ve en la figura \\ref{fig:alcance_ordenado} (p. \\pageref{fig:alcance_ordenado}).\n",
    "\n",
    "En la Gráfica de distancia de alcance se puede observar una gran cantidad de puntos están muy cerca entre sí, formando un clúster denso y bien definido.\n",
    "\n",
    "La elevación gradual de la curva muestra una transición hacia regiones cada vez menos densas, posterior a lo observado, a forma general No presenta múltiples valles profundos, esto se define, que observa un un clúster dominante principal y largo.\n",
    "\n",
    "\\subsection{Método del codo}\n",
    "\n",
    "Como se ve en la figura \\ref{fig:metodo_codo} (p. \\pageref{fig:metodo_codo}).\n",
    "\n",
    "\\begin{figure}\n",
    "    \\centering\n",
    "   \\includegraphics[width=0.95\\textwidth]{imgs/Metodo_Codo.png}\n",
    "    \\caption{Método del codo para elegir el número de grupos en el algoritmo de $K$-medias.}\n",
    "    \\label{fig:metodo_codo}\n",
    "\\end{figure}\n",
    "\n",
    "La gráfica nos indica una disminución de la inercia conforme aumenta el número de clústeres, debido a que una mayor partición de los datos permite que los puntos se agrupen en regiones más homogéneas y, por lo tanto, reduzcan su distancia al centroide asignado.\n",
    "\n",
    "El Método del Codo indica que el número óptimo de clústeres para el conjunto de datos analizado se encuentra en el rango de 4 a 5.\n",
    "\n",
    "\\subsection{Visualización t-SNE}\n",
    "\n",
    "Como se ve en la figura \\ref{fig:Agrupacion_t-sen} (p. \\pageref{fig:Agrupacion_t-sen}).\n",
    "\n",
    "\n",
    "La gráfica t-SNE muestra una dispersión amplia de puntos, esta dispersión no implica necesariamente la presencia de clústeres, por lo tanto, la ausencia de múltiples clústeres sugiere que el conjunto de datos presenta una distribución continua y con densidad separada.\n",
    "\n",
    "\\begin{figure}\n",
    "    \\centering\n",
    "   \\includegraphics[width=0.95\\textwidth]{imgs/Agrupacion_t-sen.png}\n",
    "    \\caption{Agrupacion detectadas por Optics visualizados con t-sen.}\n",
    "    \\label{fig:Agrupacion_t-sen}\n",
    "\\end{figure}\n",
    "\n",
    "\\subsection*{Resultados del Modelo LASSO}\n",
    "\n",
    "El modelo \\textbf{LASSOCV} seleccionó un valor óptimo de regularización de \n",
    "\\[\n",
    "\\alpha = 15.1469\n",
    "\\]\n",
    "\n",
    "\\subsubsection*{Métricas Obtenidas}\n",
    "\n",
    "\\begin{itemize}\n",
    "    \\item \\textbf{MAE:} 23.6632\n",
    "    \\item \\textbf{RMSE:} 42.8857\n",
    "    \\item \\textbf{R\\textsuperscript{2}:} 1.0000\n",
    "\\end{itemize}\n",
    "\n",
    "\\subsubsection*{Principales Coeficientes del Modelo LASSO}\n",
    "\n",
    "Los coeficientes más relevantes seleccionados por el modelo fueron:\n",
    "\n",
    "\\begin{table}[h!]\n",
    "\\centering\n",
    "\\begin{tabular}{lr}\n",
    "\\hline\n",
    "\\textbf{Variable} & \\textbf{Coeficiente} \\\\\n",
    "\\hline\n",
    "CNS\\_RC & 6748.569803 \\\\\n",
    "AJUSTES & 6344.221126 \\\\\n",
    "RESERVA & 4564.076995 \\\\\n",
    "GASTOS & 1078.606623 \\\\\n",
    "RVA\\_DISPONIBLE & 9.543083 \\\\\n",
    "AÑO & 0.000000 \\\\\n",
    "SUBRAMO & 0.000000 \\\\\n",
    "INCISO & -0.000000 \\\\\n",
    "MODELO & -0.000000 \\\\\n",
    "PAGO & 0.000000 \\\\\n",
    "\\hline\n",
    "\\end{tabular}\n",
    "\\end{table}\n",
    "\n",
    "El modelo asigna un peso únicamente a variables que aportan valor predictivo bajo la regularización L1, las variables con coeficiente igual a cero fueron descartadas por su baja relevancia estadística.\n",
    "\n",
    "\\section{Conclusión}\n",
    "Como se puede notar, los datos analizar solo con tienen 38 observaciones, las cuales pueden ser insuficientes para los análisis realizados, el conjunto de gráficas obtenidas durante el análisis de agrupamiento permite concluir que los datos no presentan una estructura de agrupamiento claramente definida, por consiguiente, el conjunto de datos analizado carece de particiones naturales o clústeres definidos, mostrando más bien un comportamiento continuo.\n",
    "\n",
    "\\end{document}\n"
   ]
  }
 ],
 "metadata": {
  "kernelspec": {
   "display_name": "Python 3",
   "language": "python",
   "name": "python3"
  },
  "language_info": {
   "codemirror_mode": {
    "name": "ipython",
    "version": 3
   },
   "file_extension": ".py",
   "mimetype": "text/x-python",
   "name": "python",
   "nbconvert_exporter": "python",
   "pygments_lexer": "ipython3",
   "version": "3.13.7"
  }
 },
 "nbformat": 4,
 "nbformat_minor": 5
}
